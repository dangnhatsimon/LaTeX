% FortySecondsCV LaTeX template
% Copyright © 2019-2020 René Wirnata <rene.wirnata@pandascience.net>
% Licensed under the 3-Clause BSD License. See LICENSE file for details.
%
% Please visit https://github.com/PandaScience/FortySecondsCV for the most
% recent version! For bugs or feature requests, please open a new issue on
% github.
%
% Contributors
% ------------
% * ifokkema
% * Bertbk
% * Hespe
%
% Attributions
% ------------
% * fortysecondscv is based on the twentysecondcv class by Carmine Spagnuolo
%   (cspagnuolo@unisa.it), released under the MIT license and available under
%   https://github.com/spagnuolocarmine/TwentySecondsCurriculumVitae-LaTex
% * further attributions are indicated immediately before corresponding code


%-------------------------------------------------------------------------------
%                             ADDITIONAL PACKAGES
%-------------------------------------------------------------------------------
\documentclass[
	a4paper,
	% showframes,
	% vline=2.2em,
	% maincolor=cvgreen,
	% sidecolor=gray!50,
	% sectioncolor=red,
	% subsectioncolor=orange,
	% itemtextcolor=black!80,
	% sidebarwidth=0.4\paperwidth,
	% topbottommargin=0.03\paperheight,
	% leftrightmargin=20pt,
	% profilepicsize=4.5cm,
	% profilepicborderwidth=3.5pt,
	% profilepicstyle=profilecircle,
	% profilepiczoom=1.0,
	% profilepicxshift=0mm,
	% profilepicyshift=0mm,
	% profilepicrounding=1.0cm,
]{fortysecondscv}

% improve word spacing and hyphenation
\usepackage{microtype}
\usepackage{ragged2e}

% uncomment in case you don't want any hyphenation
% \usepackage[none]{hyphenat}

% take care of proper font encoding
\ifxetexorluatex
	\usepackage{fontspec}
	\defaultfontfeatures{Ligatures=TeX}
%	\newfontfamily\headingfont[Path = fonts/]{segoeuib.ttf} % local font
\else
	\usepackage[utf8]{inputenc}
	\usepackage[T1]{fontenc}
%	\usepackage[sfdefault]{noto} % use noto google font
\fi

% enable mathematical syntax for some symbols like \varnothing
\usepackage{amssymb}

% bubble diagram configuration
\usepackage{smartdiagram}
\smartdiagramset{
	% default font size is \large, so adjust to harmonize with sidebar layout
	bubble center node font = \footnotesize,
	bubble node font = \footnotesize,
	% default: 4cm/2.5cm; make minimum diameter relative to sidebar size
	bubble center node size = 0.4\sidebartextwidth,
	bubble node size = 0.25\sidebartextwidth,
	distance center/other bubbles = 1.5em,
	% set center bubble color
	bubble center node color = maincolor!70,
	% define the list of colors usable in the diagram
	set color list = {maincolor!10, maincolor!40,
	maincolor!20, maincolor!60, maincolor!35},
	% sets the opacity at which the bubbles are shown
	bubble fill opacity = 0.8,
}


%-------------------------------------------------------------------------------
%                            PERSONAL INFORMATION
%-------------------------------------------------------------------------------
%% mandatory information
% your name
\cvname{DANG NHAT}
% job title/career
\cvjobtitle{DATA ENGINEER}

%% optional information
% profile picture
\cvprofilepic{anhCV.png}

% NOTE: ordering in sidebar will mimic the following order
% date of birth
\cvbirthday{12/04/1995}
% short address/location, use \newline if more than 1 line is required
\cvaddress{Go Vap District, HCMC}
% phone number
\cvphone{+84 0354883673}
% personal website
%\cvsite{https://pandascience.net}
% email address
\cvmail{dangnhatsimon@gmail.com}
% pgp key
%\cvkey{4096R/FF00FF00}{0xAABBCCDDFF00FF00}
% any other custom entry
%\cvcustomdata{\faFlag}{Chinese}

%-------------------------------------------------------------------------------
%                              SIDEBAR 1st PAGE
%-------------------------------------------------------------------------------
% add more profile sections to sidebar on first page
\addtofrontsidebar{
	% include gosquare national flags from https://github.com/gosquared/flags;
	% naming according to ISO 3166-1 alpha-2 country codes
	\graphicspath{{pics/flags/}}

	% social network accounts incl. proper hyperlinks
	\profilesection{Profile}
		\begin{icontable}{2.5em}{1em}
			\social{\faLinkedin}
				{https://www.linkedin.com/in/dangnhatsimon/}
				{LinkedIn}
			% \social{\faGithub}
			% 	{https://github.com/dangnhatsimon}
			% 	{Github}
		\end{icontable}

	
%	\profilesection{Hard Skills}
%		\skill{\faBalanceScale}{Sleeping almost all day}
%		\skill{\faSitemap}{Eating a lot of bamboo sprouts}
%		\skill{\faGraduationCap}{Relaxing rest of the day}


    \profilesection{Education}
    \textbf{IBM Skills Network}\\
    {Data Engineering (2022-2023).}\\
    \textbf{Google Career}\\
    {IT Automation with Python (2022-2023).}\\
    \textbf{Google Career}\\
    {Data Analytics (2022-2023).}\\
    \textbf{Obninsk Institute for Nuclear Power Engineering}\\
    {B.Sc in Nuclear Power Engineering and Thermal Physics (2014-2019).\\
    GPA: 4.6/5.0}
    }


%-------------------------------------------------------------------------------
%                              SIDEBAR 2nd PAGE
%-------------------------------------------------------------------------------
\addtobacksidebar{
	% \profilesection{About Me}
	% \aboutme{
	% 	I am a results-driven Data Engineer with a passion for creating efficient data pipelines and optimizing data workflows. My expertise lies in designing and implementing robust data warehouses, modeling data structures for optimal performance, and integrating diverse data sources. With hands-on experience in data transformation and real-time processing, I am adept at turning raw data into actionable insights.
	% }

    \profilesection{Languages}
        \begin{sidebarminipage}
            \chartlabel{Vietnamese}\\
            \chartlabel{English}\\
            \chartlabel{Russian}\\ 
            \chartlabel{Chinese}
        \end{sidebarminipage}

%	\profilesection{Diagrams}
%	\begin{sidebarminipage}
%	\chartlabel{with}
%		\chartlabel{proper}
%		\chartlabel{overflow}
%		\chartlabel{protection}
%		\chartlabel{for}
%		\chartlabel{labels}
%	\end{sidebarminipage}

%	\begin{figure}\centering
%		\smartdiagram[bubble diagram]{
%			\textcolor{white}{\textbf{Being a}} \\
%			\textcolor{white}{\textbf{Panda}}, % center bubble
%			\textcolor{black!90}{Eating},
%			\textcolor{black!90}{Sleeping},
%			\textcolor{black!90}{Rolling},
%			\textcolor{black!90}{Playing},
%			\textcolor{black!90}{Chilling}
%		}
%	\end{figure}

%	\chartlabel{Wheel Chart}
%
%	\wheelchart{3.7em}{2em}{%
%	20/3em/maincolor!50/Chill,
%	15/3em/maincolor!15/Play,
%	20/3em/maincolor!20/Eat
%	}

%	\profilesection{Barskills}
%	\barskill{\faSkyatlas}{Wearing asian rice hats}{60}
%	\barskill{\faImage}{Playing Chess}{30}
%	\barskill{\faMusic}{Playing the bamboo flute}{50}

%	\profilesection{Memberships}
%	\begin{memberships}
%		\membership[4em]{pics/logo.png}{PandaScience.net}
%		\membership[4em]{pics/logo.png}{Some longer text spanning over more than
%			only one line}
%	\end{memberships}
}


%-------------------------------------------------------------------------------
%                         TABLE ENTRIES RIGHT COLUMN
%-------------------------------------------------------------------------------
\begin{document}

\makefrontsidebar
\cvsection{Working Experience}
\begin{cvtable}[3]
	\cvitem{Oct 2022 -- Present}{Data Engineer}{PECC2}{Contribute to Big Data Development Plan for Company. Write automation script with Python, shell scripting on Linux to help company decrease working time.\\
    Monitoring health of server computers, ingest data from power plants to database server at Company. Cleaning, manipulating, processing data with 1GB/day (estimated), ensure data integrity, ethics and available to Data Scientist and ML Teams. \\
    Contributing to interactive reports and dynamic dashboard will be sent streaming to the end users to help them make decisions and improve customer experience. 
    \\Core technologies: Big Data Analytics, ETL, Data Engineering, Python, Shell, Git, Linux.}
	\cvitem{Oct 2019 -- Oct 2022}{Mechanical Piping Engineer}{PECC2}{BIM Manager and Mechanical Lead. Thermal-Mechanical and piping engineering.}
\end{cvtable}

\cvsection{Project}
\begin{cvtable}[]
    \cvitem{\color{cvsectioncolor}\href{https://github.com/dangnhatsimon/ibm}{IBM Data Engineering}}{ \\
    Implement webscraping and use APIs to extract data with Python. \\
    Create, design, \& manage relational databases \& apply database administration (DBA) concepts to RDBMSs such as MySQL, PostgreSQL, \& IBM Db2. Compose more powerful queries with advanced SQL techniques like views, transactions, stored procedures and joins. Monitor and optimize important aspects of database performance. Backing up and restoring databases, managing user roles and permissions. Troubleshoot database issues. \\
    Develop shell scripts using Linux commands, environment variables, pipes, and filters. Schedule cron jobs in Linux with crontab. \\
    Develop working knowledge of NoSQL \& Big Data using MongoDB, Cassandra, Cloudant, Hadoop, Apache Spark, Spark SQL, Spark ML, and Spark Streaming. \\
    Implement ETL \& Data Pipelines with Bash, Airflow \& Kafka; architect, populate, deploy Data Warehouses; create BI reports \& interactive dashboards. \\
    Core technologies: Data Science, ETL \& Data Pipelines, RDBMS, NoSQL and Big Data, Apache Spark, Python Programming, Data Analysis, SQL, Shell Scripting.}{}{}
    
    \cvitem{\color{cvsectioncolor}\href{https://github.com/dangnhatsimon/automation}{Automation}}{ \\
    Use Python external libraries to create and modify documents, images, and messages. \\
    Automate updating catalog information. Fetching and working with supplier data images. Process Text Files with Python Dictionaries.\\
    Using APIs to interact with web services. Uploading images, descriptions to web server. Write a script that summarizes and processes sales data into different categories. Generate a PDF report and send it through email using Python. \\ 
    Write a script to check the health status of the system. \\
    Core technologies: Bash, Linux, Python,  Shell Scripting.}{}{}

\end{cvtable}


%\cvsubsection{Study}
%\begin{cvtable}[1.5]
%	\cvitem{2006 -- 2008}{Master Studies Panda Science}{Panda Academy}
%		{Focus: Advanced rice hat studies and nouveau rain-reflecting cover
%		materials.}
%	\cvitem{}{Master Theses ($\varnothing\, 1,0$)}{Asian Rice Hat Institute}
%		{Impact of solar radiation onto rice hat cover materials with special
%		attention to water resistance.}
%	\cvitem{2003 -- 2006}{Bachelor Studies PandaScience}{Panda Academy}
%		{Focus: Bamboo morphology and its usage in different craftmanships.}
%	\cvitem{}{Bachelor Theses ($\varnothing\, 1,0$)}{Bamboo Institute}
%		{The bambo flute: An underestimated instrument in orchestras?}
%\end{cvtable}

%\cvsection{Publications}
%\begin{cvtable}
%	\cvpubitem{Cooking: 100 recipes for lazy Pandas}{Me and My Panda Friends}
%		{Panda's Culinary World}{2010}
%	\cvpubitem{Pandastasia}{Still Me}{Bamboo Books Assoc.}{2005}
%\end{cvtable}


%\cvsection{Awards}
%\begin{cvtable}
%	\cvitem{2010 -- now}{Panda of the Year}{Panda World Forum}{}
%	\cvitem{2005 -- now}{Face of World Wide Fund for Nature}{WWF}{}
%	\cvitem{2000}{Winner of Bamboo Sprouts Eating Contest}{Bamboo Society}{}
%\end{cvtable}


%\cvsection{Extra-Curricular Activities}
%\begin{cvtable}
%	\cvitemshort{Relaxing}{Master the fine art of relaxing everywhere}
%	\cvitemshort{Music}{Playing the bamboo flute in the 1st Panda Orchestra}
%	\cvitemshort{Education}{Teaching young pandas to be more panda-like}
%\end{cvtable}


\newpage
\makebacksidebar

\cvsection{Project}
\begin{cvtable}[3]  
        
    \cvitem{\color{cvsectioncolor}\href{https://github.com/dangnhatsimon/engineering}{Data Engineering}}{ \\   Ingest data from external sources and process data. Database Design.\\
    Understand Resilent Distributed Datasets, using pyspark to create a data transformation pipeline, to store data in distributed file system, transformation and actions datasets. \\
    Spark SQL for Big Data Analytics. Create, import, inspecting, cleaning, querying and saving data with pyspark. \\
    Testing data pipeline. Manage and Orchestrate Workflows with Apache Airflow, DAG schedule. \\
    Core technologies: Python, Spark, ETL, Airflow, SQL.}{}{}
    
    \cvitem{\color{cvsectioncolor}\href{https://github.com/dangnhatsimon/analytics}{Analytics}}{\\
    Mean-Variance-Standard Deviation Calculator: output the mean, variance, standard deviation, max, min, and sum of the rows, columns, and elements in matrix.\\
    Demographic Data Analyzer: analyze, manipulate data from the 1994 Census database.\\
    Medical Data Visualizer: converting, cleaning data, explore the relationship between cardiac disease, body measurements, blood markers, and lifestyle choices.\\
    Page View Time Series Visualizer: analyze and visualize a dataset containing the number of page views each day on the webpage, understand the patterns in visits and identify yearly and monthly growth.\\
    Sea Level Predictor: analyze the global average sea level change since 1880, predict the sea level change through year 2050.\\
    Combine, cleaning, manipulate, analyze and visualize to get insights from 1 millions data casual and annual membership riders. Identify trend from Cyclistic’s historical bike trip data and then planning digital media could affect marketing tactics.\\
    Combine, cleaning, manipulate, analyze the user data from FitBit Fitness Tracker to gain insights into how consumers are using the FitBit app and discover trends for Bellabeat marketing strategy.\\
    Core technologies: Data Analysis with Python by using library Numpy, Pandas, Seaborn, Matplotlib, MySQL, R, Tableau.}{}{}

\end{cvtable}

\cvsection{Skills}
\begin{cvtable}
    \cvitem{Domain}{Data Structures \& Algorithms, Probability \& Statistics.}{}{}
    \cvitem{Programming}{Python, R, Java, Scala, JavaScript, LaTeX.}{}{}
    \cvitem{Scripting}{Bash, Shell.}{}{}
    \cvitem{Database}{SQL(MySQL, PostgreSQL), NoSQL(MongoDB, Cassandra).}{}{}
    \cvitem{Big Data}{Hadoop, Spark, Kafka.}{}{}
    \cvitem{Cloud}{GCP, IBM.}{}{}
    \cvitem{Visualization}{Tableau, Power BI.}{}{}
    \cvitem{Soft Skills}{Detail-oriented, Structured \& Analytical Thinking, Teamwork, Diligent, Energentic.}{}{}
   % \chartlabel{Detail-oriented}
   % \chartlabel{Structured thinking}
   % \chartlabel{Analytical thinking}
   % \chartlabel{Hard working}
   % \\
   % \chartlabel{Teamwork}
   % \chartlabel{Data Storytelling}
\end{cvtable}

\cvsection{Certificate}
\begin{cvtable}
    \cvitem{IBM}{\href{}{IBM Data Engineering}}{}{}
    \cvitem{IBM}{\href{https://courses.cognitiveclass.ai/certificates/8c9b8d31294e425fb4a9ee7adc634829}{Big Data 101}}{}{}
    \cvitem{IBM}{\href{https://courses.cognitiveclass.ai/certificates/98e026c1d2ad48fcb99ef692d846b255}{Hadoop 101}}{}{}
    \cvitem{DataCamp}{\href{https://www.datacamp.com/completed/statement-of-accomplishment/course/41e9a0f6325b81008c8d47fe1b8021c9ebb39f21}{Building Data Engineering Pipelines in Python}}{}{}
    \cvitem{Google}{\href{https://coursera.org/share/96a1a8db90424f56a3ada058c9f1ab7c}{Google IT Automation with Python}}{}{}
    \cvitem{Google}{\href{https://coursera.org/share/81539bd07606f96b7ae9b995b37e0fa8}{Google Data Analytics}}{}{}
    \cvitem{freeCodeCamp}{\href{https://www.freecodecamp.org/certification/dangnhatsimon/data-analysis-with-python-v7}{Data Analysis with Python}}{}{}
    \cvitem{HackerRank}{\href{https://www.hackerrank.com/certificates/eab550ff20e4}{Problem Solving Basic Certificate}}{}{}
    \cvitem{HackerRank}{\href{https://www.hackerrank.com/certificates/2d7e4cc8252e}{Python Basic Certificate}}{}{}
    \cvitem{HackerRank}{\href{https://www.hackerrank.com/certificates/6687993f90b4}{SQL Advanced Certificate}}{}{}
	\cvitem{Sololearn}{\href{}{Python Intermediate}}{}{}
    \cvitem{Sololearn}{\href{https://www.sololearn.com/certificates/CC-Q4PRY1LC}{SQL Intermediate}}{}{}
\end{cvtable}
% \newgeometry{
% 	top=\topbottommargin,
% 	bottom=\topbottommargin,
% 	right=\leftrightmargin,
% 	left=\leftrightmargin
% }

%\cvsection{section}
%\cvsubsection{Subsection}
%\begin{cvtable}
%	\cvitem{<dates>}{<cv-item title>}{<location>}{<optional: description>}
%\end{cvtable}

%\cvsection{cvitem}
%\cvsubsection{Multi-line with longer description}
%\begin{cvtable}
%	\cvitem{date}{Description}{location}{Some longer and more detailed
%		description, that takes two lines of space instead of only one.}
%	\cvitem{date}{Description}{location}{Some longer and more detailed
%		description, that takes two lines of space instead of only one.}
%	\cvitem{date}{Description}{location}{Some longer and more detailed
%		description, that takes two lines of space instead of only one.}
%\end{cvtable}

%\cvsubsection{One-line without description}
%\begin{cvtable}
%	\cvitem{Award}{One-line description}{Sponsor}{}
%	\cvitem{Award}{One-line description}{Sponsor}{}
%	\cvitem{Award}{One-line description}{Sponsor}{}
%\end{cvtable}

%\cvsection{cvitemshort}
%\cvsubsection{One-line}
%\begin{cvtable}
%	\cvitemshort{Key}{Some further description}
%	\cvitemshort{Key}{Some further description}
%	\cvitemshort{Key}{Some further description}
%\end{cvtable}

%\cvsubsection{Multi-line with longer description}
%\begin{cvtable}
%	\cvitemshort{Key}{Some further description. Can fill even more than
%		only one single line while still keeping the correct indendation level.}
%	\cvitemshort{Key}{Some further description. Can fill even more than
%		only one single line while still keeping the correct indendation level.}
%	\cvitemshort{Key}{Some further description. Can fill even more than
%		only one single line while still keeping the correct indendation level.}
%\end{cvtable}

%\cvsection{cvpubitem}
%\begin{cvtable}
%	\cvpubitem{Publication title}{Authors}{Journal}{Year}
%	\cvpubitem{Publication title}{Authors}{Journal}{Year}
%	\cvpubitem{Publication title that is spanning over multiple lines and still
%		does not look too bad}{Authors}{Journal}{Year}
%\end{cvtable}

% \cvsection{References}
% \begin{cvtable}
% 	\cvpubitem{Huynh The Minh, Data Engineer}{Parcel Perform}{htminh2307@gmail.com}{Ref. 1}
% 	\cvpubitem{Nguyen Nhat Nam, Data Scientist}{CEL}{nathan.nguyennhat@gmail.com}{Ref. 2}
% \end{cvtable}
% \cvsignature

\end{document}
